It is extremely difficult to maintain consistency in any distributed database. Moreover, it becomes even more difficult for GPU-accelerated databases due to immense multi-thread environment provided by the GPU and data transfer requirements between GPU, CPU and other processing units. CoGaDB does not implement any explicit consistency standards but provides certain fault tolerance mechanism to support durability of the system which eventually leads to consistent data storage.
\newline
CoGaDB requires certain fault tolerance strategy in case of memory crunches on the GPU, since operators will not be completely processed which might lead to inconsistency in the system. CoGaDB does not support or provide any locking mechanism which helps in introducing transaction processing in the system. Due to this, CoGaDB cannot enforce any rollback mechanism in case of memory crunches. For maintaining consistency and enforcing fault tolerance standards, CoGaDB aborts the GPU operator with the memory crunch, and re-starts executing it on the CPU, where memory is available. Using this strategy, CoGaDB ensures consistency in the system, but there is no transactions support yet. 